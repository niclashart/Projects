\cleardoubleoddpage

\chapter*{Kurzfassung}
\thispagestyle{empty} %hide page numbers
Diese Arbeit beschäftigt sich mit der Entwicklung eines Retrieval-Augemented Generation (RAG)-Systems zum mehrsprachigen, technischen Informationsabruf aus Produktdatenblättern von 
Thin-Clients (Laptops) und Monitoren. Als Datengrundlage standen 1000 Produktdatenblätter zu Monitoren und 500 Produktdatenblätter zu Laptops zur Verfügung.

Das entwickelte System nutzt [verwendete Technologien, z.B. BERT, GPT, Elasticsearch] zur Verarbeitung der mehrsprachigen Dokumente und ermöglicht [spezifische Funktionen]. 
Die Evaluation zeigt [Hauptergebnisse der Tests/Messungen]. 
Die Arbeit demonstriert, dass [zentrale Erkenntnisse/Beitrag zur Wissenschaft].

Keywords: Retrieval-Augmented Generation, Mehrsprachigkeit, Produktdatenblätter, Information Retrieval, Natural Language ProcessingDas entiwckelte System verwendet